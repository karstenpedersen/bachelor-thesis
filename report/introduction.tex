\section{Introduction}\label{sec:introduction}

Over recent years, the importance of cybersecurity has grown; as societies become more dependent on technology, so do the threats and vulnerabilities that follow. As our infrastructure demands greater awareness, education, and readiness in the form of educated personnel, we also need to develop better techniques for teaching and sparking interest in different important computer science topics. To teach and engage in these complex topics in a practical way, the usage of \textit{Capture The Flag} (CTF)\cite{ctf_overview} challenges has grown and gained attraction in both educational and professional environments. In these challenges, education is gamified, leading to a dynamic teaching environment.

The purpose of this bachelor's thesis is to explore topics within computer science and cybersecurity by designing and developing five Jeopardy-style CTF challenges that introduce important concepts within topics such as web security, phishing, artificial intelligence (AI), binary exploitation (PWN), and open-source intelligence (OSINT).

We start by introducing the reader to some uncommon tools and libraries that we have used to create our challenges (Sect. \ref{sec:preliminaries}). We then discuss the design and implementation of the different challenges, starting with "Chirper", whose topic is cross-site scripting (XSS) (Sect. \ref{sec:xss-challenge}). We then look at "Maze Game", which tests the users on the Breadth-First Search (BFS) algorithm (Sect. \ref{sec:maze-challenge}). The "You've Got Mail" challenge sets up an environment where the player can execute a phishing attack (Sect. \ref{sec:mail-challenge}). The "Pwnfish" challenge presents a simple fishing game that is vulnerable to buffer overflow and format string exploits (Sect. \ref{sec:pwn-challenge}), and the "Exif Marks the Spot" challenge explores cyber hygiene through a simulated real life scenario (Sect. \ref{sec:blog-challenge}). Following the challenges, we reflect on our teamwork (Sect. \ref{sec:team-work-description}), which is followed up by a discussion of the outcome of the project (Sect. \ref{sec:discussion}) and ended with a conclusion (Sect. \ref{sec:conclusion}).
\\ \\
The source code of the challenges can be found in the attached file: \texttt{source-code.zip}.